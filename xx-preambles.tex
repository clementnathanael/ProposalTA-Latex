%==================================================================
% Ini adalah file konfigurasi
%==================================================================

%% DILARANG EDIT BAGIAN INI

% Mengatur bahasa latex
\usepackage[indonesian]{babel}
\usepackage[utf8]{inputenc}
\usepackage[T1]{fontenc}
\usepackage{geometry}

% --- Format bulan otomatis dalam Bahasa Indonesia ---
\newcommand{\bulanindonesia}{
  \ifcase\month
  \or Januari%
  \or Februari%
  \or Maret%
  \or April%
  \or Mei%
  \or Juni%
  \or Juli%
  \or Agustus%
  \or September%
  \or Oktober%
  \or November%
  \or Desember%
  \fi
}

% --- Ubah ke format kapital semua (bulan + tahun) ---
\newcommand{\bulantahun}{\MakeUppercase{\bulanindonesia\space\number\year}}



% Untuk pengaturan spacing
\usepackage{setspace}
\onehalfspacing
\usepackage[raggedrightboxes]{ragged2e}

% Untuk mengatur level section
\setcounter{secnumdepth}{6}

% Digunakan untuk memasukan gambar ke laporan.
\usepackage{graphicx}
\graphicspath{{gambar/}}
\usepackage{float}
\usepackage[hang,nooneline,scriptsize,md]{subfigure}
\usepackage[subfigure]{tocloft}

% Untuk mengatur spacing antara paragraf
\usepackage{parskip}

% Membuat indent
\usepackage{indentfirst}
\setlength\parindent{1cm}

% Untuk mengkustomisasi margin
\usepackage{scrextend}

% Untuk mengatur header dan footer
\usepackage{fancyhdr}

% Membuat seluruh tulisan menjadi Times New Roman.
\usepackage{times}

% Paket untuk font Courier (sudah benar)
\usepackage{courier}

% Paket untuk kode (sudah benar)
\usepackage{listings}

% Pengaturan untuk blok kode (gunakan 'pcr' karena ini untuk pdfLaTeX)

% Merubah numbering chapter dan section untuk judul setiap bab menggunakan romawi dan judul anak bab menggunakan arabic
\renewcommand{\thesection}{\Roman{chapter}.\arabic{section}\hspace{0.05cm}}
\renewcommand{\thesubsection}{\Roman{chapter}.\arabic{section}.\arabic{subsection}\hspace{-0.25cm}}
\renewcommand{\thesubsubsection}{\Roman{chapter}.\arabic{section}.\arabic{subsection}.\arabic{subsubsection}\hspace{-0,35cm}}

% Mengatur identasi judul section dan subsection
%\titleformat{\section}[block]{\bfseries}{\thesection.}{1em}{}
%\titleformat{\subsection}[block]{\hspace{2em}}{\thesubsection}{1em}{}

% Merubah huruf kapital pada judul daftar isi, daftar gambar, dan daftar table
\usepackage{tocloft}
\renewcommand{\cfttoctitlefont}{\hfil\large\bfseries\MakeUppercase}
\renewcommand{\cftloftitlefont}{\hfil\large\bfseries\MakeUppercase}
\renewcommand{\cftlottitlefont}{\hfil\large\bfseries\MakeUppercase}

\renewcommand{\cftdot}{.}                 % Gunakan titik sebagai penghubung
\renewcommand{\cftdotsep}{4.5}              % Jarak antar titik (semakin kecil = lebih rapat)
\renewcommand{\cftsecleader}{\cftdotfill{\cftdotsep}}
\renewcommand{\cftchapleader}{\bfseries\cftdotfill{\cftdotsep}}
\renewcommand{\cftsubsecleader}{\cftdotfill{\cftdotsep}}

% --- SOLUSI UTAMA: Mengubah format entri Bab di Daftar Isi ---
% 1. Kita definisikan ulang \thechapter menjadi Romawi. Ini standar yang baik.
\renewcommand{\thechapter}{\Roman{chapter}}

% 2. Kita definisikan ulang awalan bab di Daftar Isi.
%    Perintah ini sekarang akan mencetak "BAB " diikuti oleh \thechapter (yang sudah Romawi).
\renewcommand{\cftchappresnum}{\thechapter}

% 3. Kita beri tahu tocloft untuk TIDAK mencetak nomornya sendiri, karena kita sudah melakukannya di atas.
%    Kita lakukan ini dengan mengatur lebar kotak nomornya menjadi 0.
% Atur jarak indentasi agar semua sejajar
\setlength{\cftchapnumwidth}{1.5em}   % Lebar kolom nomor BAB (I, II, III)
\setlength{\cftsecnumwidth}{2.3em}    % Lebar kolom nomor sub-bab (I.1, I.2, dst.)
\setlength{\cftsubsecnumwidth}{3.2em} % Kalau ada subsubsection

% Samakan indent agar BAB dan sub-bab sejajar di kiri
\setlength{\cftchapindent}{0em}
\setlength{\cftsecindent}{0em}
\setlength{\cftsubsecindent}{0em}

% % --- Mengatur indentasi agar judul bab tetap lurus ---
% \newlength\mylen
% % Hitung lebar berdasarkan entri terpanjang yang mungkin (misal, "BAB VIII")
% \settowidth\mylen{\bfseries BAB VIII} % \quad menambah sedikit spasi
% \cftsetindents{chap}{0em}{\mylen}

% Mengatur font section
\usepackage{sectsty}
\sectionfont{\fontsize{12}{14}\selectfont}
\subsectionfont{\fontsize{12}{14}\selectfont}
\subsubsectionfont{\fontsize{12}{14}\selectfont}

% Untuk merupakan format penulisan BAB
\usepackage{titlesec}
\titleformat{\chapter}
{\doublespacing\fontsize{14pt}{16pt}\bfseries}
{\MakeUppercase{\chaptertitlename\ \Roman{chapter}}\filcenter}
{0.15cm}{\centering\uppercase}
\titlespacing*{\chapter}{0pt}{-1cm}{20pt}

% Mengatur spacing section
\titlespacing*{\section}
{0pt}{10pt}{0cm}
\titlespacing*{\subsection}
{0pt}{10pt}{0cm}
\titlespacing*{\subsubsection}
{0pt}{10pt}{0cm}

% Digunakan untuk mengatur caption dalam dokumen.
\usepackage[font=footnotesize,format=plain,labelfont=bf,up,textfont=up]{caption}

% Untuk menghapus titik dua (colon)
\captionsetup[figure]{labelsep=space}
\captionsetup[table]{labelsep=space}

% Mengatur nomor caption gambar
\renewcommand{\thefigure}{\Roman{chapter}.\arabic{figure}}

% Mengatur nomor caption table
\renewcommand{\thetable}{\Roman{chapter}.\arabic{table}}

% Mengatur Hyphenation pada latex
\tolerance=1
\emergencystretch=\maxdimen
\hyphenpenalty=10000
\hbadness=10000

% Untuk mengatur setting indent
\setlength\parindent{1cm}

% Untuk memasukkan table
\usepackage{tabularx}
\usepackage{multirow}

% Untuk mengatur width
\usepackage{changepage}

% Menggatur setting halaman
\usepackage{geometry}
\geometry{
left=4cm,            % <-- you want to adjust this
top=3cm,
right=3cm,
bottom=3cm,
}

% Teks testing
\usepackage{blindtext}
\usepackage{lipsum}

% Untuk mengatur subscript supscript
\usepackage{fixltx2e}

% Untuk mengatur wrap picture
\usepackage{wrapfig}

% Untuk notasi matematika
\usepackage{amsmath}
\usepackage{stmaryrd}
\usepackage{mathtools}

% untuk mengatur label nomor pada rumus
\renewcommand{\theequation}{\arabic{chapter}.\arabic{equation}}

% Untuk mengatur spacing daftar gambar
\newcommand*{\noaddvspace}{\renewcommand*{\addvspace}[1]{}}
\addtocontents{lof}{\protect\noaddvspace}

%untuk mengatur package include table in excel
% \usepackage{pgfplotstable}

% untuk mengatur landscape page
\usepackage{rotating}

% untuk list
\usepackage{enumitem}
\newenvironment{packed_enum}{
\begin{enumerate}[leftmargin=1.5\parindent]
\setlength{\itemsep}{0pt}
\setlength{\parskip}{0pt}
\setlength{\parsep}{0pt}
}{\end{enumerate}}

\newenvironment{packed_item}{
\begin{itemize}[leftmargin=1.5\parindent]
\setlength{\itemsep}{0pt}
\setlength{\parskip}{0pt}
\setlength{\parsep}{0pt}
}{\end{itemize}}

%paket untuk bibTex
\usepackage[
    backend=biber,
    authordate,
    language=english,
    autolang=other
]{biblatex-chicago}
\DefineBibliographyStrings{english}{
  and          = {dan},
  andothers    = {dkk.},
  editor       = {penyunting},
  editors      = {penyunting},
  translator   = {penerjemah},
  byeditor     = {disunting oleh},
  bytranslator = {diterjemahkan oleh},
  in           = {dalam},
  edition      = {edisi},
  pages        = {hal.},
  page         = {hal.},
  volume       = {vol.},
  number       = {no.},
  urlseen      = {diakses pada},
  url          = {tautan},
}
\addbibresource{a7-pustaka.bib}


%paket untuk mengembed kode dalam LaTeX
\usepackage{listings}
\renewcommand{\lstlistingname}{Kode}

\captionsetup[lstlisting]{
    labelformat=simple, % Menghilangkan format default yang rumit
    labelsep=space,     % Mengganti titik dua (:) dengan spasi
    format=hang         % Opsi styling agar rapi
}
\makeatletter
\lstset{
basicstyle=\fontfamily{pcr}\selectfont\small,
columns=fullflexible,
frame=single,
}

%paket untuk tabel
\usepackage{longtable}

%\setlength\LTleft{0pt}
%\setlength\LTright{0pt}
%\begin{longtable}{@{\extracolsep{\fill}}|c|c|c|@{}}

%paket untuk url
\usepackage[hidelinks]{hyperref}

% styling python
\usepackage{color}
\usepackage{listings}
\usepackage{courier}

\definecolor{mygreen}{rgb}{0,0.6,0}
\definecolor{mygray}{rgb}{0.5,0.5,0.5}
\definecolor{mymauve}{rgb}{0.58,0,0.82}

\lstset{ %
    backgroundcolor=\color{white},   % choose the background color
    basicstyle=\fontfamily{pcr}\selectfont\footnotesize, 
    breaklines=true,                 % automatic line breaking only at whitespace
    captionpos=t,                    % sets the caption-position to bottom
    commentstyle=\color{mygreen},    % comment style
    keywordstyle=\color{blue},       % keyword style
    stringstyle=\color{mymauve},     % string literal style
}

%automatic number
\usepackage{cleveref}

\crefname{figure}{gambar}{gambar}
\Crefname{figure}{Gambar}{Gambar}

\crefname{table}{tabel}{tabel}
\Crefname{table}{Tabel}{Tabel}

\crefname{equation}{persamaan}{persamaan}
\Crefname{equation}{Persamaan}{Persamaan}

%% DILARANG EDIT BAGIAN INI
