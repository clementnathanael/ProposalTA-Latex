%==================================================================
% Ini adalah lampiran
%==================================================================

%% DILARANG EDIT BAGIAN INI
\appendix
\chapter*{LAMPIRAN 1 \\ RINCIAN PERHITUNGAN AHP}
\addcontentsline{toc}{chapter}{LAMPIRAN 1 RINCIAN PERHITUNGAN AHP}
%% DILARANG EDIT BAGIAN INI

%isi lampiran kode program disini

% %% DILARANG EDIT BAGIAN INI
% \chapter*{LAMPIRAN B \\ GAMBAR-GAMBAR}
% \addcontentsline{toc}{chapter}{LAMPIRAN B GAMBAR-GAMBAR}
% %% DILARANG EDIT BAGIAN INI

%isi lampiran gambar-gambar disini

%lampiran ini dapat diedit sesuai kebutuhan, tidak harus lampiran A berisi kode program dan B gambar-gambar

\section{Rincian Perhitungan Bobot Prioritas Kriteria}
Berikut ini merupakan rincian perhitungan bobot prioritas kriteria dari AHP:
\begin{enumerate}
    \item Langkah 1: Menjumlahkan setiap kolom matriks. \\
        Setiap kolom dari matriks perbandingan (dengan nilai pecahan diubah ke desimal) dijumlahkan.
        \begin{enumerate}
            \item Jumlah Kolom K1: $1.000 + 0.333 + 0.200 + 0.143 = 1.676$
            \item Jumlah Kolom K2: $3.000 + 1.000 + 0.333 + 0.200 = 4.533$
            \item Jumlah Kolom K3: $5.000 + 3.000 + 1.000 + 0.333 = 9.333$
            \item Jumlah Kolom K4: $7.000 + 5.000 + 3.000 + 1.000 = 16.000$
        \end{enumerate}
    
    \item Langkah 2: Normalisasi matriks. \\
        Setiap elemen dalam matriks perbandingan awal (Tabel \ref{tab:matriks-kriteria}) dibagi dengan total kolomnya masing-masing. Rincian perhitungannya adalah sebagai berikut:
        \begin{enumerate}
            \item Perhitungan untuk baris K1 (Akurasi):
                \begin{enumerate}
                    \item Kolom K1: $\frac{1.000}{1.676} = 0.597$
                    \item Kolom K2: $\frac{3.000}{4.533} = 0.662$
                    \item Kolom K3: $\frac{5.000}{9.333} = 0.536$
                    \item Kolom K4: $\frac{7.000}{16.000} = 0.438$
                \end{enumerate}
            \item Perhitungan untuk baris K2 (Kelayakan TA):
                \begin{enumerate}
                    \item Kolom K1: $\frac{0.333}{1.676} = 0.199$
                    \item Kolom K2: $\frac{1.000}{4.533} = 0.221$
                    \item Kolom K3: $\frac{3.000}{9.333} = 0.321$
                    \item Kolom K4: $\frac{5.000}{16.000} = 0.313$
                \end{enumerate}
            \item Perhitungan untuk baris K3 (Skalabilitas):
                \begin{enumerate}
                    \item Kolom K1: $\frac{0.200}{1.676} = 0.119$
                    \item Kolom K2: $\frac{0.333}{4.533} = 0.073$
                    \item Kolom K3: $\frac{1.000}{9.333} = 0.107$
                    \item Kolom K4: $\frac{3.000}{16.000} = 0.188$
                \end{enumerate}
            \item Perhitungan untuk baris K4 (Efisiensi Pakar):
                \begin{enumerate}
                    \item Kolom K1: $\frac{0.143}{1.676} = 0.085$
                    \item Kolom K2: $\frac{0.200}{4.533} = 0.044$
                    \item Kolom K3: $\frac{0.333}{9.333} = 0.036$
                    \item Kolom K4: $\frac{1.000}{16.000} = 0.063$
                \end{enumerate}
        \end{enumerate}
        Hasil dari perhitungan ini kemudian disusun ke dalam Tabel \ref{tab:matriks-normalisasi-kriteria}.

        \begin{table}[H]
        \centering
        \caption{Matriks normalisasi untuk tiap kriteria}
        \label{tab:matriks-normalisasi-kriteria}
        \begin{tabular}{|l|c|c|c|c|}
        \hline
        Kriteria & K1 & K2 & K3 & K4 \\ \hline
        K1: Akurasi         & 0.597 & 0.662 & 0.536 & 0.438 \\ \hline
        K2: Kelayakan TA    & 0.199 & 0.221 & 0.321 & 0.313 \\ \hline
        K3: Skalabilitas    & 0.119 & 0.073 & 0.107 & 0.188 \\ \hline
        K4: Efisiensi Pakar & 0.085 & 0.044 & 0.036 & 0.063 \\ \hline
        \end{tabular}
        \end{table}

    \item Langkah 3: Menghitung Rata-rata Baris (Hasil Bobot Prioritas). \\
    Bobot prioritas adalah nilai rata-rata dari setiap baris pada matriks ternormalisasi.
    \begin{enumerate}
        \item Bobot K1 (Akurasi): \\ $\frac{0.597 + 0.662 + 0.536 + 0.438}{4} = 0.558 \approx \mathbf{0.55}$
        \item Bobot K2 (Kelayakan): \\ $\frac{0.199 + 0.221 + 0.321 + 0.313}{4} = 0.264 \approx \mathbf{0.26}$
        \item Bobot K3 (Skalabilitas): \\ $\frac{0.119 + 0.073 + 0.107 + 0.188}{4} = 0.122 \approx \mathbf{0.13}$
        \item Bobot K4 (Efisiensi): \\ $\frac{0.085 + 0.044 + 0.036 + 0.063}{4} = 0.057 \approx \mathbf{0.06}$
    \end{enumerate}
\end{enumerate}
Dengan demikian, didapatkan matriks hasil prioritas sebagai berikut:
    \begin{table}[h!]
    \centering
    \caption{Matriks perbandingan berpasangan dan bobot prioritas kriteria}
    \label{tab:matriks-kriteria}
    \begin{tabular}{|l|c|c|c|c||c|}
    \hline
    \textbf{Kriteria} & \textbf{K1} & \textbf{K2} & \textbf{K3} & \textbf{K4} & \textbf{Bobot Prioritas} \\ \hline
    K1: Akurasi         & 1   & 3   & 5   & 7   & 0.55 \\ \hline
    K2: Kelayakan TA    & 1/3 & 1   & 3   & 5   & 0.26 \\ \hline
    K3: Skalabilitas    & 1/5 & 1/3 & 1   & 3   & 0.13 \\ \hline
    K4: Efisiensi Pakar & 1/7 & 1/5 & 1/3 & 1   & 0.06 \\ \hline
    \end{tabular}
    \end{table}

\section{Perhitungan Evaluasi Alternatif Terhadap Setiap Kriteria}

Setiap alternatif dievaluasi untuk masing-masing kriteria melalui proses normalisasi matriks perbandingan berpasangan.
\begin{enumerate}
\item Terhadap K1: Akurasi pengetahuan
\begin{enumerate}
    \item Justifikasi: Manual paling akurat (diawasi 100\% oleh pakar). \textit{Hybrid} di urutan kedua (ada validasi). Otomatis paling tidak akurat (risiko \emph{noise}). Manual dinilai lebih akurat (7) dari otomatis dan sedikit lebih akurat (3) dari \textit{Hybrid}.
    \item Rincian Perhitungan:
        \begin{enumerate}
            \item Matriks Perbandingan (dalam desimal)
                \begin{table}[H] \centering
                \caption{Matriks Perbandingan Alternatif terhadap K1}
                \label{tab:comp-k1}
                \begin{tabular}{|l|c|c|c|} \hline
                vs K1 & Manual & \textit{Hybrid} & Otomatis \\ \hline
                Manual   & 1.000 & 3.000 & 7.000 \\
                \textit{Hybrid}  & 0.333 & 1.000 & 3.000 \\
                Otomatis & 0.143 & 0.333 & 1.000 \\ \hline
                \textbf{Jumlah} & \textbf{1.476} & \textbf{4.333} & \textbf{11.000} \\ \hline
                \end{tabular}
                \end{table}
            \item Matriks Ternormalisasi
                \begin{itemize}
                    \item Perhitungan untuk baris Manual:
                        \begin{itemize}
                            \item Kolom Manual: $\frac{1.000}{1.476} = 0.677$
                            \item Kolom \textit{Hybrid}: $\frac{3.000}{4.333} = 0.692$
                            \item Kolom Otomatis: $\frac{7.000}{11.000} = 0.636$
                        \end{itemize}
                    \item Perhitungan untuk baris \textit{Hybrid}:
                        \begin{itemize}
                            \item Kolom Manual: $\frac{0.333}{1.476} = 0.226$
                            \item Kolom \textit{Hybrid}: $\frac{1.000}{4.333} = 0.231$
                            \item Kolom Otomatis: $\frac{3.000}{11.000} = 0.273$
                        \end{itemize}
                    \item Perhitungan untuk baris Otomatis:
                        \begin{itemize}
                            \item Kolom Manual: $\frac{0.143}{1.476} = 0.097$
                            \item Kolom \textit{Hybrid}: $\frac{0.333}{4.333} = 0.077$
                            \item Kolom Otomatis: $\frac{1.000}{11.000} = 0.091$
                        \end{itemize}
                \end{itemize}
                Hasilnya disusun dalam Tabel 6.
                
                \begin{table}[H] \centering
                \caption{Matriks Ternormalisasi untuk Kriteria K1}
                \label{tab:norm-k1}
                \begin{tabular}{|l|c|c|c|} \hline
                vs K1 & Manual & \textit{Hybrid} & Otomatis \\ \hline
                Manual   & 0.677 & 0.692 & 0.636 \\
                \textit{Hybrid}   & 0.226 & 0.231 & 0.273 \\
                Otomatis & 0.097 & 0.077 & 0.091 \\ \hline
                \end{tabular}
                \end{table}
            \item Rata-rata Baris (Hasil Bobot)
                \begin{itemize}
                    \item Bobot Manual: $\frac{0.677 + 0.692 + 0.636}{3} = 0.668 \approx \mathbf{0.66}$
                    \item Bobot \textit{Hybrid}: $\frac{0.226 + 0.231 + 0.273}{3} = 0.243 \approx \mathbf{0.24}$
                    \item Bobot Otomatis: $\frac{0.097 + 0.077 + 0.091}{3} = 0.088 \approx \mathbf{0.08}$
                \end{itemize}
        \end{enumerate}
    \item Hasil Bobot: Manual (0.66), \textit{Hybrid} (0.24), Otomatis (0.08).
\end{enumerate}

\item Terhadap K2: Kelayakan implementasi Tugas Akhir
\begin{enumerate}
    \item Justifikasi: Manual paling layak (risiko teknis rendah). Otomatis dan \textit{Hybrid} tidak begitu \textit{feasible} karena memerlukan keahlian NLP yang berada di luar lingkup Tugas Akhir ini. Manual dinilai mutlak lebih layak (9) dari otomatis dan sangat lebih layak (7) dari \textit{Hybrid}.
    \item Rincian Perhitungan:
        \begin{enumerate}
             \item Matriks Perbandingan (dalam Desimal)
                \begin{table}[H] \centering
                \caption{Matriks Perbandingan Alternatif terhadap K2}
                \label{tab:comp-k2}
                \begin{tabular}{|l|c|c|c|} \hline
                vs K2 & Manual & \textit{Hybrid} & Otomatis \\ \hline
                Manual   & 1.000 & 7.000 & 9.000 \\
                \textit{Hybrid}   & 0.143 & 1.000 & 2.000 \\
                Otomatis & 0.111 & 0.500 & 1.000 \\ \hline
                \textbf{Jumlah} & \textbf{1.254} & \textbf{8.500} & \textbf{12.000} \\ \hline
                \end{tabular}
                \end{table}
            \item Matriks Ternormalisasi
                 \begin{itemize}
                    \item Perhitungan untuk baris Manual:
                        \begin{itemize}
                            \item Kolom Manual: $\frac{1.000}{1.254} = 0.797$
                            \item Kolom \textit{Hybrid}: $\frac{7.000}{8.500} = 0.824$
                            \item Kolom Otomatis: $\frac{9.000}{12.000} = 0.750$
                        \end{itemize}
                    \item Perhitungan untuk baris \textit{Hybrid}:
                        \begin{itemize}
                            \item Kolom Manual: $\frac{0.143}{1.254} = 0.114$
                            \item Kolom \textit{Hybrid}: $\frac{1.000}{8.500} = 0.118$
                            \item Kolom Otomatis: $\frac{2.000}{12.000} = 0.167$
                        \end{itemize}
                    \item Perhitungan untuk baris Otomatis:
                        \begin{itemize}
                            \item Kolom Manual: $\frac{0.111}{1.254} = 0.089$
                            \item Kolom \textit{Hybrid}: $\frac{0.500}{8.500} = 0.059$
                            \item Kolom Otomatis: $\frac{1.000}{12.000} = 0.083$
                        \end{itemize}
                \end{itemize}
                Hasilnya disusun dalam Tabel 8.
                \begin{table}[H] \centering
                \caption{Matriks Ternormalisasi untuk Kriteria K2}
                \label{tab:norm-k2}
                \begin{tabular}{|l|c|c|c|} \hline
                vs K2 & Manual & Hybrid & Otomatis \\ \hline
                Manual   & 0.797 & 0.824 & 0.750 \\
                Hybrid   & 0.114 & 0.118 & 0.167 \\
                Otomatis & 0.089 & 0.059 & 0.083 \\ \hline
                \end{tabular}
                \end{table}
            \item Rata-rata Baris (Hasil Bobot)
                \begin{itemize}
                    \item Bobot Manual: $\frac{0.797 + 0.824 + 0.750}{3} = 0.790 \approx \mathbf{0.79}$
                    \item Bobot Hybrid: $\frac{0.114 + 0.118 + 0.167}{3} = 0.133 \approx \mathbf{0.13}$
                    \item Bobot Otomatis: $\frac{0.089 + 0.059 + 0.083}{3} = 0.077 \approx \mathbf{0.10}$
                \end{itemize}
        \end{enumerate}
    \item Hasil Bobot: Manual (0.79), \textit{Hybrid} (0.13), Otomatis (0.10).
\end{enumerate}

\item Terhadap K3: Skalabilitas
\begin{enumerate}
    \item Justifikasi: Otomatis paling skalabel karena dapat memproses ribuan dokumen. Hybrid di urutan kedua, dan Manual sangat tidak skalabel. Berdasarkan penilaian, Otomatis dinilai sangat lebih skalabel (7) dari Manual, dan sedikit lebih skalabel (3) dari Hybrid. Selain itu, Hybrid juga dinilai sedikit lebih skalabel (3) dari Manual.
    \item Rincian Perhitungan:
        \begin{enumerate}
            \item Matriks Perbandingan (dalam Desimal)
                \begin{table}[H] \centering
                \caption{Matriks Perbandingan Alternatif terhadap K3}
                \label{tab:comp-k3}
                \begin{tabular}{|l|c|c|c|} \hline
                vs K3 & Manual & Hybrid & Otomatis \\ \hline
                Manual   & 1.000 & 0.333 & 0.143 \\
                Hybrid   & 3.000 & 1.000 & 0.333 \\
                Otomatis & 7.000 & 3.000 & 1.000 \\ \hline
                \textbf{Jumlah} & \textbf{11.000} & \textbf{4.333} & \textbf{1.476} \\ \hline
                \end{tabular}
                \end{table}
            \item Matriks Ternormalisasi
                \begin{itemize}
                    \item Perhitungan untuk baris Manual:
                        \begin{itemize}
                            \item Kolom Manual: $\frac{1.000}{11.000} = 0.091$
                            \item Kolom Hybrid: $\frac{0.333}{4.333} = 0.077$
                            \item Kolom Otomatis: $\frac{0.143}{1.476} = 0.097$
                        \end{itemize}
                    \item Perhitungan untuk baris Hybrid:
                        \begin{itemize}
                            \item Kolom Manual: $\frac{3.000}{11.000} = 0.273$
                            \item Kolom Hybrid: $\frac{1.000}{4.333} = 0.231$
                            \item Kolom Otomatis: $\frac{0.333}{1.476} = 0.226$
                        \end{itemize}
                    \item Perhitungan untuk baris Otomatis:
                        \begin{itemize}
                            \item Kolom Manual: $\frac{7.000}{11.000} = 0.636$
                            \item Kolom Hybrid: $\frac{3.000}{4.333} = 0.692$
                            \item Kolom Otomatis: $\frac{1.000}{1.476} = 0.677$
                        \end{itemize}
                \end{itemize}
                Hasilnya disusun dalam Tabel 10.
                \begin{table}[H] \centering
                \caption{Matriks Ternormalisasi untuk Kriteria K3}
                \label{tab:norm-k3}
                \begin{tabular}{|l|c|c|c|} \hline
                vs K3 & Manual & Hybrid & Otomatis \\ \hline
                Manual   & 0.091 & 0.077 & 0.097 \\
                Hybrid   & 0.273 & 0.231 & 0.226 \\
                Otomatis & 0.636 & 0.692 & 0.677 \\ \hline
                \end{tabular}
                \end{table}
            \item Rata-rata Baris (Hasil Bobot)
                \begin{itemize}
                    \item Bobot Otomatis: $\frac{0.636 + 0.692 + 0.677}{3} = 0.668 \approx \mathbf{0.65}$
                    \item Bobot Hybrid: $\frac{0.273 + 0.231 + 0.226}{3} = 0.243 \approx \mathbf{0.25}$
                    \item Bobot Manual: $\frac{0.091 + 0.077 + 0.097}{3} = 0.088 \approx \mathbf{0.10}$
                \end{itemize}
        \end{enumerate}
    \item Hasil Bobot: Otomatis (0.65), \textit{Hybrid} (0.25), Manual (0.10).
\end{enumerate}
\item Terhadap K4: Efisiensi keterlibatan pakar
\begin{enumerate}
    \item Justifikasi: Otomatis paling efisien (intervensi pakar minimal). \textit{Hybrid} membutuhkan waktu untuk validasi. Manual paling boros karena membutuhkan waktu lama dari pakar. Otomatis dinilai sangat lebih efisien (7) dari manual, dan sedikit lebih efisien (3) dari Hybrid.
    \item Rincian Perhitungan:
        \begin{enumerate}
            \item Matriks Perbandingan (dalam Desimal)
                \begin{table}[H] \centering
                \caption{Matriks Perbandingan Alternatif terhadap K4}
                \label{tab:comp-k4}
                \begin{tabular}{|l|c|c|c|} \hline
                vs K4 & Manual & Hybrid & Otomatis \\ \hline
                Manual   & 1.000 & 0.333 & 0.143 \\
                Hybrid   & 3.000 & 1.000 & 0.333 \\
                Otomatis & 7.000 & 3.000 & 1.000 \\ \hline
                \textbf{Jumlah} & \textbf{11.000} & \textbf{4.333} & \textbf{1.476} \\ \hline
                \end{tabular}
                \end{table}
            \item Matriks Ternormalisasi
                \begin{itemize}
                    \item Perhitungan untuk baris Manual:
                        \begin{itemize}
                            \item Kolom Manual: $\frac{1.000}{11.000} = 0.091$
                            \item Kolom Hybrid: $\frac{0.333}{4.333} = 0.077$
                            \item Kolom Otomatis: $\frac{0.143}{1.476} = 0.097$
                        \end{itemize}
                    \item Perhitungan untuk baris Hybrid:
                        \begin{itemize}
                            \item Kolom Manual: $\frac{3.000}{11.000} = 0.273$
                            \item Kolom Hybrid: $\frac{1.000}{4.333} = 0.231$
                            \item Kolom Otomatis: $\frac{0.333}{1.476} = 0.226$
                        \end{itemize}
                    \item Perhitungan untuk baris Otomatis:
                        \begin{itemize}
                            \item Kolom Manual: $\frac{7.000}{11.000} = 0.636$
                            \item Kolom Hybrid: $\frac{3.000}{4.333} = 0.692$
                            \item Kolom Otomatis: $\frac{1.000}{1.476} = 0.677$
                        \end{itemize}
                \end{itemize}
                Hasilnya disusun dalam Tabel 12.
                \begin{table}[H] \centering
                \caption{Matriks Ternormalisasi untuk Kriteria K4}
                \label{tab:norm-k4}
                \begin{tabular}{|l|c|c|c|} \hline
                vs K4 & Manual & Hybrid & Otomatis \\ \hline
                Manual   & 0.091 & 0.077 & 0.097 \\
                Hybrid   & 0.273 & 0.231 & 0.226 \\
                Otomatis & 0.636 & 0.692 & 0.677 \\ \hline
                \end{tabular}
                \end{table}
            \item Rata-rata Baris (Hasil Bobot)
                \begin{itemize}
                    \item Bobot Otomatis: $\frac{0.636 + 0.692 + 0.677}{3} = 0.668 \approx \mathbf{0.62}$
                    \item Bobot Hybrid: $\frac{0.273 + 0.231 + 0.226}{3} = 0.243 \approx \mathbf{0.27}$
                    \item Bobot Manual: $\frac{0.091 + 0.077 + 0.097}{3} = 0.088 \approx \mathbf{0.11}$
                \end{itemize}
        \end{enumerate}
    \item Hasil Bobot: Otomatis (0.62), \textit{Hybrid} (0.27), Manual (0.11).
\end{enumerate}
\end{enumerate}
\end{enumerate}